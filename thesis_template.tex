%%%%%%%%%%%%%%%%%%%%%%%%%%%%%%%%%%%%%%%%%%%%%%%%%%%%%%%%%%%%%%%%%%%%%%%%%%%%
% For the thesis there is no need for any special style file.              %
% This shows how to do the title page and handles the contents page,       %
% chapter headings, numbering of theorems, bibliographic references, etc.  %
% Note that to get references right it will normally be necessary          %
% to run LaTeX twice.                                                      %
%%%%%%%%%%%%%%%%%%%%%%%%%%%%%%%%%%%%%%%%%%%%%%%%%%%%%%%%%%%%%%%%%%%%%%%%%%%%

\documentclass{report}
\usepackage{amsmath}                                                            
\usepackage{amsfonts}

%\usepackage[notcite,notref]{showkeys} 
%Use the showkeys package to view labels. For draft version only!

                                                           
\setlength{\textwidth}{5.5in} 
\setlength{\textheight}{8.5in}
\topmargin 1.5in 
\oddsidemargin 0.3in 
\evensidemargin 0.3in

\begin{document}

\begin{titlepage}\phantom{|}\vspace{-0.75in}

\begin{center}
    \underline{Pricing and profit testing of life insurance}
\end{center}

\vspace{1.5in}%{2.0in}

\begin{center}
    Thesis submitted at the University of Leicester \\
    in partial fulfilment of the requirements for \\ 
    the MSc degree in Financial Mathematics and Computation \\
\end{center}

\vspace{.5in}

\begin{center}
    by
\end{center}

\vspace{.5in}

\begin{center}
    Lemin Wu\\
    Department of Mathematics \\
    University of Leicester \\
\end{center}

\vspace{0.5in}

\begin{center}
    September 2012
\end{center}

\end{titlepage}







\topmargin 0in
\pagenumbering{roman}
\tableofcontents

\global\baselineskip24pt plus2pt minus2pt



\newtheorem{ttt}{THEOREM}[chapter]  %Theorem: use "\begin{ttt} ... \end{ttt}"
\newtheorem{lll}[ttt]{LEMMA}        %Lemma: use "\begin{lll} ... \end{lll}" 
\newtheorem{ccc}[ttt]{COROLLARY}                                                
\newtheorem{ppp}[ttt]{PROPOSITION}                                              
\newtheorem{conj}[ttt]{CONJECTURE}  %Conjecture: "\begin{conj} ... \end{conj}"
\newtheorem{rmdef}[ttt]{DEFINITION} %Definition: "\begin{ddd} ... \end{ddd}"
\newtheorem{rmexa}[ttt]{EXAMPLE}    %Example: "\begin{eee} ... \end{eee}" 
\newtheorem{rmrem}[ttt]{REMARK}     %Remark: "\begin{rrr} ... \end{rrr}" 
                                     
\newenvironment{ddd}{\begin{rmdef}\rm}{\end{rmdef}}                             
\newenvironment{eee}{\begin{rmexa}\rm}{\end{rmexa}}                             
\newenvironment{rrr}{\begin{rmrem}\rm}{\end{rmrem}}                             
                                              
					                                        
\newenvironment{pf}[1][Proof]{                                                  
\par\noindent{\em #1}. }{\hfill\framebox(6,6)\par\medskip}                      
                                                                               
% use \begin{pf} ...Your proof here... \end{pf}
% or \begin{pf}[Proof of the lemma]  ... \end{pf}"
% or \begin{pf}[Proof of Theorem \ref{mytheorem}]  ... \end{pf}"





\chapter*{Declaration}                % The * means no number for this chapter
\addcontentsline{toc}{chapter}{\hspace{0.2in}Declaration}
All sentences or passages quoted in this project dissertation from other
people's work have been specifically acknowledged by clear cross referencing
to author, work and page(s).  I understand that failure to do this amounts
to plagiarism and will be considered grounds for failure in this module and
the degree examination as a whole.

\bigskip

\noindent
Name:

Lemin Wu\\


\bigskip

\noindent
Signed:


\bigskip

\noindent
Date:


\chapter*{Abstract}
\addcontentsline{toc}{chapter}{\hspace{0.2in}Abstract}

A summary of the thesis in about 200 words.









\chapter*{Introduction}

\pagenumbering{arabic}
\addcontentsline{toc}{chapter}{\hspace{0.2in}Background}

In this thesis we consider the work of Gauss \cite[ch 2]{bib:Gauss}
and Hilbert \cite{bib:Hilbert1} on the subject of the title.



1. why doing the project\\
2. what have I done\\
3. what have I achieve\\
4. what work can be done in extension\\










\chapter*{Convensions}

The following variables are used though out the project

$\ddot{a}$\\
$\ddot{A}$\\
$\ddot{A}^{(m)}_{[x]}$\\
$\ddot{a}^{(m)}_{[x]}$\\

\section*{Notation}
















\section*{assurance?annuity}
\addcontentsline{toc}{section}{\hspace{0.2in}Gauss's Work}

We will discuss Gauss's work in Chapter \ref{chapter-on-Gauss}.

\section*{unit-linked}
\addcontentsline{toc}{section}{\hspace{0.2in}Hilbert's Work}

We will discuss Hilbert's work in Chapter \ref{chapter-on-Hilbert}.










\chapter*{Background}



?Do I need background? mathematical background








\chapter{Results of Gauss}    \label{chapter-on-Gauss}

In \cite{bib:Gauss} Gauss proved the following very important result.

\begin{ttt}[Gauss]    \label{Gauss'sTheorem}
Some very profound result.
\end{ttt}


Later on in Chapter \ref{chapter-on-Hilbert}
we will have more to say about Theorem \ref{Gauss'sTheorem}.


\section{Gauss's youthful work}




\section{Gauss's mature work}













\chapter{Unit-linked insurance}     \label{unit-linked}

%In \cite{bib:Hilbert1} Hilbert considered these questions from a
%more abstract point of view. He proved the following result.

%\begin{ttt}[Hilbert]      \label{Hilbert'sTheorem}
%Some even more profound result.
%\end{ttt}

%In Chapter \ref{chapter-on-Gauss}
%a special case of Theorem \ref{Hilbert'sTheorem} was proved.
%We can prove an even more general result.

%\begin{ttt}    \label{mytheorem}
%An extremely profound result.
%\end{ttt}

%\begin{pf} As any fool can plainly see, it's true! 
%\end{pf}


%In the previous chapter we used different method to obtain mortality rate\\
In this chapter we introduce the unit-linked insurance contract. We start from some assumptions to establish a deterministic pricing model, and demonstrate that the deterministic pricing test is not accurate enough for this contract since the uncertainty in the investment return is not diversifiable.  

To solve this problem, we consider to use a stochastic pricing test with future investment return as an random variable, then testify the stochastic test will determine a better premium and reserve. 














\begin{thebibliography}{99}             %Any other two digit number will do
\addcontentsline{toc}{chapter}{\hspace{0.2in}Bibliography}

\bibitem{bib:Gauss} Gauss, C.F.,
    ``Disquisitiones Arithmeticae'', Leipzig, 1801.

\bibitem{bib:Hilbert1}  Hilbert, D., 
    {\em \"{U}ber tern\"{a}re definite Formen},
    Acta Math., {\bf 17} (1893), 169--197.

\end{thebibliography}


\chapter*{Appendix 1: Optional Extra}
\addcontentsline{toc}{chapter}{\hspace{0.2in}Appendix 1: Optional Extra}

\begin{center}
   {\Large A PROGRAM TO COMPUTE EIGENVALUES}
\end{center}

\begin{verbatim}    
100 GOTO 200
200 END
\end{verbatim}    

\chapter*{Appendix 2: More Extra}
\addcontentsline{toc}{chapter}{\hspace{0.2in}Appendix 2: More Extra}


%
% Symbol Entry for Math Symbol Tables
%
\newcommand{\X}[1]{$#1$&\texttt{\string#1}\hspace*{1ex}}
\newsavebox{\symbbox}
\newenvironment{symbols}[1]%
{\par\vspace*{2ex}
\begin{lrbox}{\symbbox}
\hspace*{4ex}\begin{tabular}{@{}#1@{}}}%
{\end{tabular}\end{lrbox}\makebox[\textwidth]{\usebox{\symbbox}}\par\medskip}



\begin{table}[!h]
\caption{Lowercase Greek Letters.}
\begin{symbols}{*4{cl}}
 \X{\alpha}     & \X{\theta}     & \X{o}          & \X{\upsilon}  \\
 \X{\beta}      & \X{\vartheta}  & \X{\pi}        & \X{\phi}      \\
 \X{\gamma}     & \X{\iota}      & \X{\varpi}     & \X{\varphi}   \\
 \X{\delta}     & \X{\kappa}     & \X{\rho}       & \X{\chi}      \\
 \X{\epsilon}   & \X{\lambda}    & \X{\varrho}    & \X{\psi}      \\
 \X{\varepsilon}& \X{\mu}        & \X{\sigma}     & \X{\omega}    \\
 \X{\zeta}      & \X{\nu}        & \X{\varsigma}  & &             \\
 \X{\eta}       & \X{\xi}        & \X{\tau} 
\end{symbols}
\end{table}

\begin{table}[!h]
\caption{Uppercase Greek Letters.}
\begin{symbols}{*4{cl}}
 \X{\Gamma}     & \X{\Lambda}    & \X{\Sigma}     & \X{\Psi}      \\
 \X{\Delta}     & \X{\Xi}        & \X{\Upsilon}   & \X{\Omega}    \\
 \X{\Theta}     & \X{\Pi}        & \X{\Phi} 
\end{symbols}
\end{table}


\begin{table}[!tbp]
\caption{Math Alphabets.}
\begin{symbols}{@{}*3l@{}}
Example& Command &Required package\\
\hline
\rule{0pt}{1.05em}$\mathrm{ABCDE abcde 1234}$
        & \verb|\mathrm{ABCDE abcde 1234}|
        &       \\
$\mathit{ABCDE abcde 1234}$
        & \verb|\mathit{ABCDE abcde 1234}|
        &       \\
$\mathnormal{ABCDE abcde 1234}$
        & \verb|\mathnormal{ABCDE abcde 1234}|
        &  \\
$\mathcal{ABCDE abcde 1234}$
        & \verb|\mathcal{ABCDE abcde 1234}|
        &  \\
$\mathfrak{ABCDE abcde 1234}$
        & \verb|\mathfrak{ABCDE abcde 1234}|
        &\textsf{amsfonts}  or \textsf{amssymb}  \\
$\mathbb{ABCDE abcde 1234}$
        & \verb|\mathbb{ABCDE abcde 1234}|
        &\textsf{amsfonts}  or \textsf{amssymb} \\
\end{symbols}
\end{table}



\end{document} 
 
