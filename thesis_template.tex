%%%%%%%%%%%%%%%%%%%%%%%%%%%%%%%%%%%%%%%%%%%%%%%%%%%%%%%%%%%%%%%%%%%%%%%%%%%%
% For the thesis there is no need for any special style file.              %
% This shows how to do the title page and handles the contents page,       %
% chapter headings, numbering of theorems, bibliographic references, etc.  %
% Note that to get references right it will normally be necessary          %
% to run LaTeX twice.                                                      %
%%%%%%%%%%%%%%%%%%%%%%%%%%%%%%%%%%%%%%%%%%%%%%%%%%%%%%%%%%%%%%%%%%%%%%%%%%%%

\documentclass{report}
\usepackage{amsmath}                                                            
\usepackage{amsfonts}
\usepackage{booktabs}

%\usepackage[notcite,notref]{showkeys} 
%Use the showkeys package to view labels. For draft version only!

                                                           
\setlength{\textwidth}{5.5in} 
\setlength{\textheight}{8.5in}
\topmargin 1.5in 
\oddsidemargin 0.3in 
\evensidemargin 0.3in

\begin{document}

\begin{titlepage}\phantom{|}\vspace{-0.75in}

\begin{center}
    \underline{Pricing and profit testing of life insurance}
\end{center}

\vspace{1.5in}%{2.0in}

\begin{center}
    Thesis submitted at the University of Leicester \\
    in partial fulfilment of the requirements for \\ 
    the MSc degree in Financial Mathematics and Computation \\
\end{center}

\vspace{.5in}

\begin{center}
    by
\end{center}

\vspace{.5in}

\begin{center}
    Lemin Wu\\
    Department of Mathematics \\
    University of Leicester \\
\end{center}

\vspace{0.5in}

\begin{center}
    September 2012
\end{center}

\end{titlepage}







\topmargin 0in
\pagenumbering{roman}
\tableofcontents

\global\baselineskip24pt plus2pt minus2pt



\newtheorem{ttt}{THEOREM}[chapter]  %Theorem: use "\begin{ttt} ... \end{ttt}"
\newtheorem{lll}[ttt]{LEMMA}        %Lemma: use "\begin{lll} ... \end{lll}" 
\newtheorem{ccc}[ttt]{COROLLARY}                                                
\newtheorem{ppp}[ttt]{PROPOSITION}                                              
\newtheorem{conj}[ttt]{CONJECTURE}  %Conjecture: "\begin{conj} ... \end{conj}"
\newtheorem{rmdef}[ttt]{DEFINITION} %Definition: "\begin{ddd} ... \end{ddd}"
\newtheorem{rmexa}[ttt]{EXAMPLE}    %Example: "\begin{eee} ... \end{eee}" 
\newtheorem{rmrem}[ttt]{REMARK}     %Remark: "\begin{rrr} ... \end{rrr}" 
                                     
\newenvironment{ddd}{\begin{rmdef}\rm}{\end{rmdef}}                             
\newenvironment{eee}{\begin{rmexa}\rm}{\end{rmexa}}                             
\newenvironment{rrr}{\begin{rmrem}\rm}{\end{rmrem}}                             
                                              
					                                        
\newenvironment{pf}[1][Proof]{                                                  
\par\noindent{\em #1}. }{\hfill\framebox(6,6)\par\medskip}                      
                                                                               
% use \begin{pf} ...Your proof here... \end{pf}
% or \begin{pf}[Proof of the lemma]  ... \end{pf}"
% or \begin{pf}[Proof of Theorem \ref{mytheorem}]  ... \end{pf}"





\chapter*{Declaration}                % The * means no number for this chapter
\addcontentsline{toc}{chapter}{\hspace{0.2in}Declaration}
All sentences or passages quoted in this project dissertation from other
people's work have been specifically acknowledged by clear cross referencing
to author, work and page(s).  I understand that failure to do this amounts
to plagiarism and will be considered grounds for failure in this module and
the degree examination as a whole.

\bigskip

\noindent
Name:

Lemin Wu\\


\bigskip

\noindent
Signed:


\bigskip

\noindent
Date:


\chapter*{Abstract}
\addcontentsline{toc}{chapter}{\hspace{0.2in}Abstract}

A summary of the thesis in about 200 words.









\chapter*{Introduction}

\pagenumbering{arabic}
\addcontentsline{toc}{chapter}{\hspace{0.2in}Background}

In this thesis we consider the work of Gauss \cite[ch 2]{bib:Gauss}
and Hilbert \cite{bib:Hilbert1} on the subject of the title.



1. why doing the project\\
2. what have I done\\
3. what have I achieve\\
4. what work can be done in extension\\










\chapter*{Convensions}

The following variables are used though out the project

$\ddot{a}$\\
$\ddot{A}$\\
$\ddot{A}^{(m)}_{[x]}$\\
$\ddot{a}^{(m)}_{[x]}$\\

\section*{Notation}
















\section*{assurance?annuity}
\addcontentsline{toc}{section}{\hspace{0.2in}Gauss's Work}

\section*{unit-linked}
\addcontentsline{toc}{section}{\hspace{0.2in}Hilbert's Work}











\chapter*{Background}



?Do I need background? mathematical background








\chapter{Results of Gauss}    \label{annuity}


\begin{ttt}[Gauss]    \label{Gauss'sTheorem}
Some very profound result.
\end{ttt}


we will have more to say about Theorem \ref{Gauss'sTheorem}.


\section{Gauss's youthful work}




\section{Gauss's mature work}













\chapter{Unit-linked insurance}     \label{unit-linked}

%In \cite{bib:Hilbert1} Hilbert considered these questions from a
%more abstract point of view. He proved the following result.

%\begin{ttt}[Hilbert]      \label{Hilbert'sTheorem}
%Some even more profound result.
%\end{ttt}

%In Chapter \ref{chapter-on-Gauss}
%a special case of Theorem \ref{Hilbert'sTheorem} was proved.
%We can prove an even more general result.

%\begin{ttt}    \label{mytheorem}
%An extremely profound result.
%\end{ttt}

%\begin{pf} As any fool can plainly see, it's true! 
%\end{pf}


%In the previous chapter we used different method to obtain mortality rate

In this chapter we introduce the unit-linked insurance contract. We start from some assumptions to establish a deterministic pricing model, and demonstrate that the deterministic pricing test is not accurate enough for this contract, which is caused by the uncertainty, or in other words risk, in the investment is not diversifiable.  

To solve this problem, we consider to use a stochastic pricing test with future investment return as an random variable, then testify the stochastic test will determine a better premium and reserve. 


\section{Background}

In some countries unit-linked contract is also called equity-linked contract, it is because the single or regular premium paid by the policy holder will be invested on units of Equity or bonds on their own behalf.


Once the insurer receive the premium, it will be payed into policy holder's fund after deduction of expenses and management costs, and the deduction goes to insurer's account. In UK, there is an un-allocated percentage, which is another agreed regular deduction from policy holder's fund to insurer's account.

 \cite{bib:GMMB}If the policy holder survival to the maturity date, he/she will receive the greater value of total premiums payed in or the fund in policy holder's account. This is called guranteed minimum maturity benefit (GMMB). But the if the death happens first, there is a guaranteed minimum death benefit, which allows policy holder's estate to receive the policy holder's fund with an extra amount. 


\subsection{Some assumptions *** change variables!!} 

Before we establish the model, we need some assumptions, the following are some ideas from  \cite{bib:unitlinkeg} book {\em Actuarial Mathematics for Life Contingent Risks,} and some online sources \cite{bib:unitlinkegonline}. 


A company is going to issue a new 10 years unit-linked contract to people from 55 to 60 years old. In the contract, it stated that the annual premium is \textsterling 5200, with an un-allocated rate of 5\% in the first year and 1\% in the subsequent years. Expenses occurs the same time as the premium payed in, which is 13\% (including commision) and 0.7\% repectively. At the end of each year, there is a further management charge of 0.8\% of policy holder's fund. All the above deductions from policy holder's account will be transferred to insurer's account.

When the contract mature, the policy holder can received the greater of policy holder's fund and accumulation of premiums payed in. But if the death happens before the mature date, policy holder's estate will receive 110\% of the policy holder's fund where the extra 10\% is from insurer's account.

The insurance company is also prepared to have 10\% of policy holders surrender the contract in the first year and 5\% in the second year, but no more in the subsequent years. The policy holders who surrender the contract will received their premium(s) after deduction of management cost at the end of the year.

In the Chapter \ref{annuity}, we had introduce different methods to calculate the mortality rate, but here we are going to use a constant force of mortality, $\mu_x=2.2218$, for people aged 55 to 60, which means the mortality rate of policy holders $q_x=0.006$ is a constant in each year. The reason we set the mortality rate to a constant will be explained later in this chapter.  

BABE!!! HOW CAN I MAKR A EQUATION AT THE BOTTOM OF THE PAGE? ??



\section{Deterministic pricing and reserving}

Last section we introduce the unit-linked contract and have some assumptions to help us establish the model. But here is something we did not count in, the uncertainty in the investment return in both policy holder's account ans insurer's account. 

For deterministic pricing, we used conservative interest rates for both accounts, 8\% for policy holder's account and 5\% for insurer's account, and assume no reserve holds in this case. Since the mortality rate is a constant, as well as the conservative interest rates, the cash flow for different policy holders will be exactly the same. Now we set one as an example.

\subsection{Cash flow analysis}

Now from all the assumptions, we can create the cash flow tables for both policy holder and insurer's account for the  10 years term. First let us explore the policy holder's account. 



\begin{figure}[h]
    \begin{tabular}{p{1cm} p{1.5cm} p{2cm} p{1.5cm} p{1cm} p{1.5cm} p{2cm} p{1.5cm} }
\toprule
\multicolumn{8}{c}{Cash flow for policy holder's account} \\
\cmidrule(r){3-7}

Year t & Annual premium & Allocated premium & Fund brought forward & Interest & Fund at time $t^-$ & Management cost & Fund bring forward \\
\midrule
1	&5200	&4940	&0&	395.2	&5335.2	&42.68	&5292.52\\
2	&5200	&5148	&5292.52	 &   835.24	&11275.76 &	90.21	&11185.55\\
3	&5200	&5148	&11185.55&  1306.68	&17640.24	&141.12  &17499.12\\
4&	5200	&5148	&17499.12&  1811.77&	24458.89	&    195.67       &24263.21\\
5&	5200	&5148&	24263.21	&    2352.89&	31764.11	&   254.11	&31509.99\\
6&	5200	&5148&	31509.99	&   2932.64&	39590.64	&   316.73	&39273.91\\
7&	5200	&5148&	39273.91	&   3553.75&	47975.67	&   383.81	&47591.86\\
8&	5200	&5148&	47591.86	&   4219.189&	56959.05	&  455.67	    &56503.38\\
9&	5200	&5148&	56503.38	&   4932.11&	66583.49	&  532.67 	&66050.82\\
10&	5200	&5148	&66050.82	&5695.91	&76894.73	&615.16	&76279.57\\
\bottomrule
\end{tabular}
\end{figure}




%\begin{table}[h]
%\caption{Cash flow in policy holder's account} % title of Table
%\centering  % used for centering table
%\begin{tabular}{c c c c c c c c c c} % centered columns (4 columns)
%\hline                     %inserts double horizontal lines
%Year t & Ann& A & Fear & it & F at $t^-$ & mst & Fuear (\$) \\ [0.5ex] % inserts table 
%heading
%\hline                  % inserts single horizontal line
%1   & 5200    &    4940  &	0  &	395.2   &	5384.6  & 43.0768	&5341.5232 \\
%2	&5200	&5148	&5341.52	   &  835.24	   &11433.58	     & 91.47   	&11342.11\\
%3	&5200	&5148	&11342.11	  &1306.68	    &17974.22	   &143.80	    &17830.43\\
%4	&5200	&5148	&17830.43	  &1811.77	  & 25046.49	   &200.37	   &24846.11\\
%5	&5200	&5148	&24846.11	  &2352.90  	&32693.58   &261.55	   &32432.04\\
%6	&5200	&5148	&32432.04   &	3382.20   &	40962.24	&   327.70 	&40634.54\\
%7	&5200	&5148	&40634.54	   &4120.43   	&49902.97	   &399.22	 &  49503.75\\
%8	&5200	&5148	&49503.75   &4918.66   	&59570.40	   &476.56   	&59093.84\\
%9	&5200	&5148	&59093.84	   &5781.77   	&70023.60   &	560.19	  & 69463.42\\
%10	&5200	&5148	&69463.42&	6715.03   	&81326.44   &650.61	   &80675.83\\ [1ex]      % [1ex] %adds vertical space
%\hline 

%\end{tabular}
%\label{table:nonlin} % is used to refer this table in the text
%\end{table}




Here are the method used to calculate the table above from colum to column.  (BABE, I DON'T MEAN METHOD HERE, WHAT WORD CAN I USE?)

\textbf{Allocated premium:} For the first year, the premium allocated in the policy holder's fund will be (1-5\%) of the premium payed, where 5\% is the agreed un-allocated premium which will be payed into insurer's account. For the subsequent years, the un-allocated rate decrease to 1\%, hence there will be 99\% premium payed into policy holder's fund.

BABE!! CAN YOU PLEASE LINE UP THE TWO EQUATIONS BELOW? THX!! XXX

First year allocated premium:                5,200 $\times$ 0.95= 4,940

Subsequent years allocated premium:   5,200 $\times$ 0.99= 5,148

\textbf{Interest:} It is the profit policy holder obtain from the the accumulation of premium payed in at the beginning of the year and fund brought forward from end of last year with the assumed interest rate 8\%.

eg.  year t=5, interest = (5,148 + 24263.21) $\times$ 0.08 = 2,352.89

\textbf{Fund at time $t^-$:} It is the accumulation of premium payed in at the beginning of the year, fund brought forward from end of last year and interest earned.

\textbf{Management cost:} In our assumption, the management cost is 0.08\% of the policy holder's fund, which is the fund at time $t^-$ here. 

\textbf{Fund bring forward:} It is the fund left in policy holder's account after deduction of management cost. 


Table SIMON!!!! HOW TO NUMBER THE TABLE???!? is showing the insurer's account cash flow for one policy holder. 


\begin{figure}[ht]
\hfill
\begin{tabular}{c c c c c c c c c c}
\toprule
\multicolumn{9}{c}{Cash flow for insurer's account} \\
\cmidrule(r){3-7}

Year t & Annual premium & Un-allocated premium & Expenses & Interest &Management & Expected death benefit & Profit  \\
\midrule

0&0&0&676&0&0&0&-676&-676\\
1&5200&260&36.4&11.18&42.686&3.18&274.29&274.29\\
2&5200&52&36.4&0.78&90.21&6.71&99.87&89.35\\
3&5200&52&36.4&0.78&141.12&10.50&147.00&124.14\\
4&5200&52&36.4&0.78&195.67&14.56&197.49&165.78\\
5&5200&52&36.4&0.78&254.11&18.91&251.59&209.92\\
6&5200&52&36.4&0.78&316.73&23.56&309.54&256.73\\
7&5200&52&36.4&0.78&383.81&28.56&371.63&306.38\\
8&5200&52&36.4&0.78&455.67&33.90&438.15&359.05\\
9&5200&52&36.4&0.78&532.67&39.63&509.42&414.95\\
10&5200&52&36.4&0.78&615.16&45.77&585.77&474.28\\

\bottomrule
\end{tabular}
\end{figure}


In insurer's account, un-allocated premium, expense, interest and management cost are calculated in the same way as in policy holder's account. You may notice there is an extra row in the table, year 0 in insurer's cash flow, this is because the expenses will always occurs before receive the premium, eg. renting, stationary cost and employees' salary. 

For expected death benefit, profit and profit signature are calculated as below:

\textbf{Expected death benefit:} If the death occurs before the contract mature, the policy holder's estate will receive the fund from policy holder's account with extra 10\% from insurer account. The death benefit we calculate here is the 10\% from insurer's account. 

At time 0, there policy holder just enter the contract, so the probability in force is 1.

In year 1, policy holders pay the first premium right after sign the contract, so the probability in force is still 1. 

During the first year, by our assumption, there will be 10\% policy holders surender the contract, and the mortality rate for the other 90\% is 0.006. Therefore, at the beginning of the second calandar year, insurer will be expecting $p_2=$ (1-0.006) $\times$(1-0.1) =0.8946 of policy holders renew their contracts. 

Same as in the first year, the insurer is expecting $p_3$= (1-0.006-0.5) $\times$$p_2$= 0.8445024 will renew the contract, since by assumption there are 5\% policy holder will surender during the second year.

From the third year, the probability in force will be $p_t$ = (1-0.006) $\times$ $p_{t-1}$. (Showing in table !!!SIMON ADD TABLE NUMBER FOR ME!!)


\begin{figure}[h]
\begin{tabular}{|l|l|}
  \hline
  \multicolumn{2}{|c|}{Survival Rate} \\
  \hline
t=	&probability in force\\
0	&1\\
1	&1\\
2	&0.8946\\
3	&0.8445024\\
4	&0.839435386\\
5	&0.834398773\\
6	&0.829392381\\
7	&0.824416026\\
8	&0.81946953\\
9	&0.814552713\\
10	&0.809665397\\
  \hline
\end{tabular}
\end{figure}


\textbf{Profit:} For each year, profit is calculated by the fomula

Profit = Un-allocated premium - Expenses + Interest + Management cost - Death benefit

\textbf{Profit signature:} This is profit the insurer can make with probality the contracts are still in force, which is defined as $\Pi_t = p_x \times$ Profit$_t$


\subsection{Profitability}


\section{Stochastic pricing}
\subsection{time series and stochastic process}
\subsection{Monte Carlo simulation}
\subsection{Pricing}
\subsection{Reserving}



\section{Deterministic VS Stochastic test}






\begin{thebibliography}{99}             %Any other two digit number will do
\addcontentsline{toc}{chapter}{\hspace{0.2in}Bibliography}

\bibitem{bib:Gauss} Gauss, C.F.,
    ``Disquisitiones Arithmeticae'', Leipzig, 1801.

\bibitem{bib:GMMB}  David C. M. Dickson, Mary R. Hardy and Howard R. Waters,
    {\em Actuarial Mathematics for Life Contingent Risks,},
    3rd edition, 2011, 
    page 374-375.

\bibitem{bib:unitlinkeg}  David C. M. Dickson, Mary R. Hardy and Howard R. Waters,
    {\em Actuarial Mathematics for Life Contingent Risks,},
    3rd edition, 2011, 
    page 375.


%\bibitem{bib:unitlinkeg}  Hans U. Gerber,  James C. Hickman, Donald A. Jones and Cecil J. Nesbitt,
%   {\em Actuarial Mathematics,},
%  3rd edition, 2011, 
%    page 375.

\bibitem{bib:unitlinkegonline}  Mrs Giselle du Toit
    {\em Actuarial Mathematics for Life Contingent Risks,},
    2nd edition, 2011, 
    page 375.




\end{thebibliography}


\chapter*{Appendix 1: Optional Extra}
\addcontentsline{toc}{chapter}{\hspace{0.2in}Appendix 1: Optional Extra}

\begin{center}
   {\Large A PROGRAM TO COMPUTE EIGENVALUES}
\end{center}

\begin{verbatim}    
100 GOTO 200
200 END
\end{verbatim}    

\chapter*{Appendix 2: More Extra}
\addcontentsline{toc}{chapter}{\hspace{0.2in}Appendix 2: More Extra}


%
% Symbol Entry for Math Symbol Tables
%
\newcommand{\X}[1]{$#1$&\texttt{\string#1}\hspace*{1ex}}
\newsavebox{\symbbox}
\newenvironment{symbols}[1]%
{\par\vspace*{2ex}
\begin{lrbox}{\symbbox}
\hspace*{4ex}\begin{tabular}{@{}#1@{}}}%
{\end{tabular}\end{lrbox}\makebox[\textwidth]{\usebox{\symbbox}}\par\medskip}



\begin{table}[!h]
\caption{Lowercase Greek Letters.}
\begin{symbols}{*4{cl}}
 \X{\alpha}     & \X{\theta}     & \X{o}          & \X{\upsilon}  \\
 \X{\beta}      & \X{\vartheta}  & \X{\pi}        & \X{\phi}      \\
 \X{\gamma}     & \X{\iota}      & \X{\varpi}     & \X{\varphi}   \\
 \X{\delta}     & \X{\kappa}     & \X{\rho}       & \X{\chi}      \\
 \X{\epsilon}   & \X{\lambda}    & \X{\varrho}    & \X{\psi}      \\
 \X{\varepsilon}& \X{\mu}        & \X{\sigma}     & \X{\omega}    \\
 \X{\zeta}      & \X{\nu}        & \X{\varsigma}  & &             \\
 \X{\eta}       & \X{\xi}        & \X{\tau} 
\end{symbols}
\end{table}

\begin{table}[!h]
\caption{Uppercase Greek Letters.}
\begin{symbols}{*4{cl}}
 \X{\Gamma}     & \X{\Lambda}    & \X{\Sigma}     & \X{\Psi}      \\
 \X{\Delta}     & \X{\Xi}        & \X{\Upsilon}   & \X{\Omega}    \\
 \X{\Theta}     & \X{\Pi}        & \X{\Phi} 
\end{symbols}
\end{table}


\begin{table}[!tbp]
\caption{Math Alphabets.}
\begin{symbols}{@{}*3l@{}}
Example& Command &Required package\\
\hline
\rule{0pt}{1.05em}$\mathrm{ABCDE abcde 1234}$
        & \verb|\mathrm{ABCDE abcde 1234}|
        &       \\
$\mathit{ABCDE abcde 1234}$
        & \verb|\mathit{ABCDE abcde 1234}|
        &       \\
$\mathnormal{ABCDE abcde 1234}$
        & \verb|\mathnormal{ABCDE abcde 1234}|
        &  \\
$\mathcal{ABCDE abcde 1234}$
        & \verb|\mathcal{ABCDE abcde 1234}|
        &  \\
$\mathfrak{ABCDE abcde 1234}$
        & \verb|\mathfrak{ABCDE abcde 1234}|
        &\textsf{amsfonts}  or \textsf{amssymb}  \\
$\mathbb{ABCDE abcde 1234}$
        & \verb|\mathbb{ABCDE abcde 1234}|
        &\textsf{amsfonts}  or \textsf{amssymb} \\
\end{symbols}
\end{table}



\end{document} 
 
